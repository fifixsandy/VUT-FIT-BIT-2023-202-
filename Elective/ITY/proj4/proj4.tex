%%%%%%%%%%%%%%%%%%%%%%%%%%%%%%%%%%%%%%%%%%%%%%%%%%%%%%%%%%%%%%%%%%%%%%%%%%%
% MENO:             ITY Projekt 4 - Bibliografické citace
% AUTOR:            Filip Novak
% XLOGIN:           xnovakf00
% DATUM:            15-04-2024
% POSLEDNA UPRAVA:  17-04-2024
%%%%%%%%%%%%%%%%%%%%%%%%%%%%%%%%%%%%%%%%%%%%%%%%%%%%%%%%%%%%%%%%%%%%%%%%%%%

\documentclass[a4paper, 11pt]{article}
\usepackage[left=2cm, top=3cm, text={17cm,24cm}]{geometry}
\usepackage[utf8]{inputenc}
\usepackage[T1]{fontenc}
\usepackage{times}
\usepackage[english,czech,slovak]{babel}
\usepackage{csquotes}
\usepackage[backend=biber,style=iso-numeric]{biblatex}
\usepackage{hyperref}

\addbibresource{odkazy.bib}
\begin{document}

\begin{titlepage}
    \begin{center}
        \textsc{\Huge{Vysoké učení technické v~Brně\\[0.4em]} % názov školy a fakulty ponechávam český
        \huge{Fakulta informačních technologií}}\\        
        \vspace{\stretch{0.382}}
        \LARGE{Typografia a~publikovanie\,--\,4. projekt}\\
        \Huge{Bibliografické citácie}\\
        \vspace{\stretch{0.618}}
    \end{center}
    \Large{\today\hfill Filip Novák}
\end{titlepage}

\begin{center}
    \Huge{\textbf{Fonty a~typografia}}
\end{center}

\section*{Úvod}
Fonty, alebo znakové sady sú súhrnom všetkých znakov jedného typu a~rezu \cite{Rybicka_latex}. Sú jednou z~najzákladnejších častí typografie. Vhodný výber a~správne zásady použitia, či kombinovania
fontov pridávajú dielu hodnotu nielen po estetickej stránke, ale aj dokazujú profesionálnosť autora a~dôveryhodnosť práce.
Preto by malo byť v~záujme všetkých autorov mať tieto znalosti a~orientovať sa v~danej problematike.

\section{Výber softvéru pre sadzbu}
Pri výbere softvéru je potrebné stanoviť si požiadavky, ako sú napríklad jednoduchosť použitia, rozmanitosť funkcií,
prenositeľnosť, či možnosti prispôsobenia. Pre jednoduché texty sa využívajú tzv. \emph{WYSIWYG} textové procesory (What You See Is What You Get),
ktoré zobrazujú výsledný dokument v~reálnom čase \cite{Lukes_diploma}. Ak je jednoduchosť použitia druhoradá a~dôležité je vydolovať zo softvéru čo najviac,
využívajú sa rôzne sádzacie systémy, primárne \LaTeX \cite{Lukes_diploma, Found_latex}. Ten bol pôvodne vytvorený pre sadzbu
matematických či fyzikálnych odborných textov, ponúka však veľa možností ako upravovať a~prispôsovať fonty a~písmo ako celok (viz \cite{Helmut_latex}, kapitola 4).

\section{Vlastnosti fontu}
Medzi základné vlastnosti fontu patria hrúbka ťahu (pomer hrúbky ťahu voči veľkosti písma), sklon písma (uhol naklonenia) a~spôsob
zakončenia ťahu (patkové alebo bezpatkové písmo) \cite{Sladovnikova_bakalar}.
Špeciálna úprava fontu môže byť použitá pre zvýrazňovanie častí textu s~účelom upozornenia čitateľa. Tradične sa
využíva \emph{kurzíva}, \textsc{kapitálky}, VERZÁLKY alebo \textbf{polohrubé písmo} \cite{Rybicka_latex}.

\section{Zlaté pravidlá písma}
\emph{Konzistencia} je v~typografii jedno z~pravidiel, ktoré by sa malo dodržiavať za každých podmienok \cite{Twelve_golden_rules}. Použitie veľa rôznych spôsobov pre zvýrazňovanie
podstatných častí textu pôsobí rušivo nielen po estetickej stránke, ale môže aj zmiasť čitateľa. Náhla zmena fontu v~strede textu bez opodstatnenia
nedáva zmysel a~je len na škodu.
\emph{Efekty} môžu prácu ozvláštniť a~pridať estetickú hodnotu, no nie je vítané používať ich za každú cenu. Prebytok
tieňov, rozmazaní a~farieb vedie k~tomu, že dielo bude považované za amatérske. 
V~neposlednom rade je \emph{čitateľnosť}. Platí, že hlavným účelom akejkoľvek literárnej práce je jasné a~zrozumiteľné podanie informácií čitateľovi. Experimentálne fonty, kde nie je jasné, aký znak
je aké písmeno zabraňuje rýchlemu pochopeniu textu.

Pravidlo čitateľnosti má však výnimku vo veľmi špeciálnom prípade \cite{Oppenheimer_hard_fonts}, kedy autor chce prinútiť
konzumenta zamyslieť sa nad textom a~rozlúštiť ho. Prieskum ukázal, že žiaci, ktorí sa učili z~textu používajúceho menej čitateľný font si zapamätali
viac, ako žiaci, ktorí text dostali v~čitateľnejšej podobe.

\section{Prekvapivé dopady výberu písma}
Písmo a~konkrétne fonty v~nás môžu vyvolávať rôzne emócie \cite{Sladovnikova_bakalar}. Fonty využívajúce zaoblenia na nás pôsobia ukľudňujúco v~porovnaní s~fontami, ktoré sú ostrejšie a~špicatejšie. Ostré hrany si spájame s~nebezpečnými situáciami a~preto sú vhodné pre varovania, či upozornenia.
Zdobené písma v~nás často evokujú pocit kreativity, detskej radosti a~nevinnosti. Niektoré fonty sú špeciálne navrhnuté pre ľudí s~dyslexiou \cite{Dyslexie}. Každé písmeno je upravené, aby sa ľuďom s~touto poruchou najlepšie čítalo.  

Pri výbere písma pre náš účel by sme teda mali myslieť na tieto fakty, aby sme dosiahli očakávaný výsledok.

\section{Problém kontroly}
Pre aspekty diel, kde je presnosť dôležitá, existuje množstvo automatických alebo poloautomatických nástrojov na kontrolu korektnosti. Pre matematické vzorce v~\LaTeX u je to napríklad
\emph{EqFix} \cite{DSET}, ktorý kontroluje a~opravuje chybné matematické rovnice na základe dodaných príkladov.

Takýto nástroj pre overovanie, ktorý by s~veľkou presnosťou určil, či výber fontov a~efektov je absolútne korektný, by sa vyvíjal veľmi ťažko. Správnosť dokáže určiť len človek, ktorý pozná zámer diela a~je oboznámený s~pravidlami a~zásadami.

\section{Fonty dnes}
Aj keď typografia ako taká už existuje dlhú dobu, nové fonty sú stále navrhované a~pridávané do obehu. Vo svete sa každoročne organizujú workshopy a~konferencie o~písme a~typografii, ktoré nadšencom
ponúkajú možnosť stretnúť sa s~odborníkmi a~naučiť sa novým znalostiam. Jedna z~konferencií v~Českej republike nesie názov \emph{Brno Bold Typo} \cite{Brno_bold_typo} a~organizuje ju štúdio Bloomfield.
Usporiadavajú sa aj súťaže, kde sa podľa rôznych kritérií vyberajú tie najlepšie fonty. V~roku 2023 spoločnosť Fiverr po dlhej analýze za výherný font vyhlásilo Montserrat 2024 \cite{Fiverr}.

\section*{Záver}
Keďže fonty do veľkej miery zasahujú do celkového dojmu z~diela, každý autor by mal byť schopný korektne používať prostriedky,
ktoré sú mu ponúknuté. Táto oblasť typografie sa stále rozvíja a~je prospešné byť v~obraze s~aktuálnymi trendami.
\newpage
\printbibliography
\end{document}

% V proj4.log sa zobrazovala chyba overfull, pretože názov zborníka bol príliš dlhý - slovo Software som nahradil skratkou SW.
% Varovanie '\mainlang' is deprecated in favour of '\textmainlang' je pravdepodobne z bibtex štýlu, použil som ten z prezentácie.
% Varovanie destination with the same identifier je (podľa toho, čo som našiel na internete) spôsobené balíkom hyperref, ktorý je (mám pocit) 
% konfliktný s takmer všetkým. Keďže v predchádzajúcich projektoch sa varovanie spôsobené hyperref nijak nebralo do úvahy, ponechám balík.
